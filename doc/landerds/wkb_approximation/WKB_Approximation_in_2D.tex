% \documentclass[aps,secnumarabic,nobalancelastpage,amsmath,amssymb,
% nofootinbib]{revtex4}

\documentclass[aps,twocolumn,secnumarabic,nobalancelastpage,amsmath,amssymb,
nofootinbib]{revtex4}

\usepackage{graphics}      % standard graphics specifications
\usepackage{graphicx}      % alternative graphics specifications
\usepackage{longtable}     % helps with long table options
\usepackage{url}           % for on-line citations
\usepackage{bm}            % special 'bold-math' package


\begin{document}
\title{WKB Approximation in Two Dimensions}
\author         {Daniel S. Landers}
\email          {dans.landers@gmail.com}
%\homepage       {http://web.mit.edu/fac/www}
\affiliation    {SUNYPI Physics}
\date{\today}

\begin{abstract}
(abstract)
\end{abstract}

\maketitle
%%%%%%%%%%%%%%%%%%%%%%%%%%%%%%%%%%%%%%%%%%%%%%%%%%%%%%%%%%%%%%%%%%%%%%%%%%%%%
\section{Separation of Variables in 2D Cartesian Coordinates}
In two dimensions (2D) the time independent Schr\"odinger equation becomes
\begin{align}
\nonumber
-&\,\frac{\hbar^2}{2m}
\left( \frac{\partial^2}{\partial x^2} + \frac{\partial^2}{\partial y^2}\right)
\psi(x,y) + V(x,y)\psi(x,y)&&\\ = &\,E\psi(x,y),&&
\label{eq:schro_2d}
\end{align}
%
where $V(x,y)$ is a 2D potential. We'll assume that the solution is the
product of separate functions in x and y:
\begin{equation}
\psi(x,y) = X(x)Y(y).
\label{eq:psixy}
\end{equation}
%
The 2D Schr\"odinger equation can be separated into two ordinary differential
equations (ODE) if the potential can be written as a sum of x and y
components:
\begin{equation}
V(x,y) = V_x(x) + V_y(y).
\label{eq:Vxy}
\end{equation}
%
Substituting these definitions from equations~(\ref{eq:psixy},\ref{eq:Vxy})
into the 2D Schr\"odinger equation (\ref{eq:schro_2d}), we have
\begin{align}
\nonumber
&-\frac{\hbar^2}{2m}
\left( \frac{\partial^2}{\partial x^2}
+ \frac{\partial^2}{\partial y^2}\right)X(x)Y(y)&&\\
&+ \left(V_x(x) + V_y(y)\right)X(x)Y(y) = EX(x)Y(y).&&
\label{eq:schro_1}
\end{align}
%
Then, distributing the derivatives and the separated potential results in
\begin{align}
\nonumber
&\left[-\frac{\hbar^2}{2m}Y(y)\frac{d^2X(x)}{dx^2}+V_x(x)X(x)Y(y)\right]&&\\
\nonumber
+ &\left[-\frac{\hbar^2}{2m}X(x)\frac{d^2Y(y)}{dy^2}+V_y(x)X(x)Y(y)\right]&&\\
= & \,EX(x)Y(y).&&
\label{eq:schro_2}
\end{align}
%
And after dividing by $X(x)Y(y)$, we are left with
\begin{align}
\nonumber
E=&\left[-\frac{\hbar^2}{2m}\frac{1}{X(x)}\frac{d^2X(x)}{dx^2}+V_x(x)\right]&&\\
+&\left[-\frac{\hbar^2}{2m}\frac{1}{Y(y)}\frac{d^2Y(y)}{dy^2}+V_y(x)\right].&&
\label{eq:schro_3}
\end{align}
%
The first term is a function of $x$ only, while the second term depends only
on $y$. Also, $E$ is constant over $x$ and $y$. Therefore the first two terms
are each a constant:
\begin{subequations}
\label{eq:schro_sep}
\begin{equation}
E_x = -\frac{\hbar^2}{2m}\frac{1}{X(x)}\frac{d^2X(x)}{dx^2}+V_x(x)
\label{subeq:schro_sep_x}
\end{equation}
\begin{equation}
E_y = -\frac{\hbar^2}{2m}\frac{1}{Y(y)}\frac{d^2Y(y)}{dy^2}+V_y(x)
\label{subeq:schro_sep_y}
\end{equation}
\begin{equation}
E=E_x+E_y
\label{subeq:schro_sep_E}
\end{equation}
\end{subequations}
%%%%%%%%%%%%%%%%%%%%%%%%%%%%%%%%%%%%%%%%%%%%%%%%%%%%%%%%%%%%%%%%%%%%%%%%%%%%
\section{Example 1: 2D Finite Barrier in an Infinite Square Well}
We can imagine a 2D system, that has a finite barrier in both x and y as
shown in figure <fig 1> as its potential. The total potential is the sum of
the two:
\begin{equation}
V(x,y) = V_x(x) + V_y(y)
\label{eq:potential_fin_barr}
\end{equation}
We'll start by finding the wavefunctions in the x-dimension.

\subsection{$E<V_0$}
When the total energy is less than the potential barrier the wavefunction
is sinusiodal on either side. Within the barrier it exponentially decays
approaching the center.

%%%%%%%%%%%%%%%%%%%%%%%%%%%%%%%%%%%%%%%%%%%%%%%%%%%%%%%%%%%%%%%%%%%%%%%%%%%%
\subsubsection{Region $\alpha$ $(-d_1 < x < 0)$}
In region $alpha$, where $V=0$, the Schr\"odinger equation becomes
\begin{equation}
\frac{d^2}{dx^2}\psi(x) + \frac{2mE}{\hbar^2}\psi(x) = 0,
\label{eq:schro_lt_V0_reg1}
\end{equation}
%
and the general solution is
\begin{subequations}
\label{eq:sol_lt_V0_reg1}
\begin{equation}
\psi(x) = A e^{ikx} + A_0 e^{-ikx},
\label{subeq:sol_lt_V0_reg1_a}
\end{equation}
\text{where}
\begin{equation}
k=\sqrt{2mE}/\hbar.
\label{subeq:sol_lt_V0_reg1_b}
\end{equation}
\end{subequations}
%
However at $x=-d_1$, $|\psi(x)^2|=0$. Therefore,
\begin{equation}
A_0 = \frac{-A e^{ikd_1}}{e^{-ikd_1}} = -A e^{2ikd_1}.
\label{eq:A0_defn}
\end{equation}
%
Finally, the wavefunction in region $\alpha$ is
\begin{equation}
\psi_\alpha(x)=A[e^{ikx} - e^{2ikd_1}e^{-ikx}]
\label{eq:psi_lt_V0_reg1}
\end{equation}

%%%%%%%%%%%%%%%%%%%%%%%%%%%%%%%%%%%%%%%%%%%%%%%%%%%%%%%%%%%%%%%%%%%%%%%%%%%
\subsubsection{Region $\beta$ $(0 < x < d_2)$}
In region $\beta$, $V(x)=V_0$ and the Schr\"odinger equation becomes
\begin{equation}
\frac{d^2}{dx^2}\psi(x) - \frac{2m}{\hbar^2}(V_0-E)\psi(x) = 0,
\label{eq:schro_lt_V0_reg2}
\end{equation}
%
and the general solution is
\begin{subequations}
\label{eq:sol_lt_V0_reg2}
\begin{equation}
\psi_\beta(x)=Be^{nx} + Ce^{-nx},
\end{equation}
\text{where}
\begin{equation}
n=\frac{\sqrt{2m(V_0-E)}}{\hbar}
\end{equation}
\end{subequations}

%%%%%%%%%%%%%%%%%%%%%%%%%%%%%%%%%%%%%%%%%%%%%%%%%%%%%%%%%%%%%%%%%%%%%%%%%%%
\subsubsection{Region $\gamma$ $(d_2 < x < d_2+d_3)$}
In region $\gamma$ where $V=0$, the general solution is the same as (\ref{eq:sol_lt_V0_reg1}), but we'll designate different constants:
\begin{subequations}
\label{eq:sol_lt_V0_reg3}
\begin{equation}
\psi_\gamma(x) = D e^{ikx} + D_0 e^{-ikx},
\end{equation}
\text{where again}
\begin{equation}
k=\sqrt{2mE}/\hbar.
\end{equation}
\end{subequations}
%
At the right boundary of the well, the wavefunction must be zero: $|\psi(d_2+d_3)|^2=0$. Thus
\begin{equation}
D_0 = -D \frac{e^{ik(d_2+d_3)}}{e^{-ik(d_2+d_3)}} = -D e^{2ik(d_2+d_3)},
\label{eq:D0_defn}
\end{equation}
%
and finally the wavefunction becomes
\begin{equation}
\psi_\gamma(x)=D[e^{ikx} - e^{2ik(d_2+d_3)} e^{-ikx}].
\label{eq:psi_lt_V0_reg3}
\end{equation}
%%%%%%%%%%%%%%%%%%%%%%%%%%%%%%%%%%%%%%%%%%%%%%%%%%%%%%%%%%%%%%%%%%%%%%%%%%%
\subsubsection{Boundary Conditions Between Regions}
Additional boundary conditions can be imposed between the three regions. This
will lead to the eigenvalues and eventually, to solving for the constants.
The wavefunction and its derivative must be continuous at the boundaries.

Equating the the wavefunctions at the boundary of the $\alpha$ region and
$\beta$ region:
%\begin{equation}
%\psi_\alpha(x)\rvert_{x=0} = \psi_\beta(x)\rvert_{x=0}
%\end{equation}
%
\begin{equation}
A[e^{ikx} - e^{2ikd_1}e^{-ikx}]\rvert_{x=0} = [Be^{nx} + Ce^{-nx}]\rvert_{x=0},
\end{equation}
thus
\begin{equation}
A[1 - e^{2ikd_1}] = B + C.
\label{eq:alpha_beta_bound}
\end{equation}
%
And equating the derivatives at the boundary:
\begin{equation}
A[ike^{ikx} + ike^{2ikd_1}e^{-ikx}]\rvert_{x=0} =
[Bne^{nx} - Cne^{-nx}]\rvert_{x=0},
\end{equation}
therefore,
\begin{equation}
iAk[1 + e^{2ikd_1}] = Bn - Cn.
\label{eq:d_alpha_beta_bound}
\end{equation}
%
Equating the wavefunctions at the boundary of the $\beta$-region and the
$\gamma$-region $(x=d_2)$, leads to
\begin{equation}
Be^{nd_2} + Ce^{-nd_2} = D[e^{ikd_2} - e^{2ik(d_2+d_3)} e^{-ikd_2}].
\label{eq:beta_gamma_bound}
\end{equation}
%
And by equating the derivatives between the $\beta$- and $\gamma$-regions,
\begin{equation}
Bne^{nd_2} - Cne^{-nd_2} = D[ike^{ikd_2} + ike^{2ik(d_2+d_3)} e^{-ikd_2}].
\label{eq:d_beta_gamma_bound}
\end{equation}
%
We can combine equations (\ref{eq:alpha_beta_bound},
\ref{eq:d_alpha_beta_bound}, \ref{eq:beta_gamma_bound},
\ref{eq:d_beta_gamma_bound}) into a matrix equation:
\begin{subequations}
\begin{equation}
\nonumber
\bf{M}\begin{pmatrix} A\\ B\\ C\\ D\\ \end{pmatrix} = 0\\
\end{equation}
\begin{equation}
=\begin{pmatrix}
1 - e^{2ikd_1}  &  -1  &  -1  &  0 \\
ik[1 + e^{2ikd_1}]  &  -n  &  n  &  0 \\
0  &  e^{nd_2}  &  e^{-nd_2}  &  e^{ikd_2} - f \\
0  &  ne^{nd_2}  &  -ne^{-nd_2}  &  ik[e^{ikd_2} + f] \\
\end{pmatrix}\begin{pmatrix} A\\ B\\ C\\ D\\ \end{pmatrix}
\end{equation}
\text{where}
\begin{equation}
\label{eq:f_term}
f = e^{2ik(d_2+d_3)} e^{-ikd_2}.
\end{equation}
\end{subequations}
%
And then the Eigenvalues can be found from the following determinant:
\begin{equation}
\begin{vmatrix}
1 - e^{2ikd_1}  &  -1  &  -1  &  0 \\
ik[1 + e^{2ikd_1}]  &  -n  &  n  &  0 \\
0  &  e^{nd_2}  &  e^{-nd_2}  &  e^{ikd_2} - f \\
0  &  ne^{nd_2}  &  -ne^{-nd_2}  &  ik[e^{ikd_2} + f] \\
\end{vmatrix} = 0.
\end{equation}
%
Evaluating the determinant, and grouping like terms leaves us with
\begin{align}
\nonumber
(1-e^{2ikd_1})\big[&-ikn(e^{nd_2}+e^{-nd_2})(e^{ikd_2} + f)&&\\
\nonumber
&+ n^2(e^{nd_2}-e^{-nd_2})(e^{ikd_2} - f)\big]&&\\
\nonumber
+\,ik(1+e^{2ikd_1})\big[&ik(e^{-nd_2}-e^{nd_2})(e^{ikd_2} + f)&&\\
\nonumber
&+n(e^{nd_2}+e^{-nd_2})(e^{ikd_2} - f)\big]=0.&&\\
\end{align}
%
After substituting in equation (\ref{eq:f_term}), and replacing some
exponentials with trigonometric and hyperbolic functions, we have
\begin{align}
\nonumber
\tan(kd_1)\big[&-2ikn(\cosh(nd_2))[e^{2ikd_2}+e^{2ik(d_2+d_3)}]&&\\
\nonumber
&+2n^2 \sinh(nd_2)[e^{2ikd_2}-e^{2ik(d_2+d_3)}]\big]&&\\
\nonumber
+k\big[&-2ik\sinh(nd_2)[e^{2ikd_2}+e^{2ik(d_2+d_3)}]&&\\
\nonumber
&+2n\cosh(nd_2)[e^{2ikd_2}-e^{2ik(d_2+d_3)}]\big] = 0.&&\\
\end{align}
%
By dividing by $e^{2ikd_2}$, and factoring out the $(1-e^{2ikd_3})$ and
$i(1+e^{2ikd_3})$ terms:
\begin{align}
\nonumber
i&(1+e^{2ikd_3})\big[2kn\cosh(nd_2)\tan(kd_1)+2k^2\sinh(nd_2)\big]&&\\
\nonumber
=\,&(1-e^{2ikd_3})\big[2n^2\sinh(nd_2)\tan(k_d1)+2nk\cosh(nd_2)\big].&&\\
\end{align}
%
The quotient of the aforementioned terms is $\tan(kd_3)$, and we can divide
by $\cosh(nd_2)$. Finally, the condition for eigenvalues when $E<V_0$ is
\begin{align}
\nonumber
&kn\tan(kd_1)+k^2\tanh(nd_2)&&\\
\nonumber
=&\tan(kd_3)\big[n^2\tanh(nd_2)\tan(kd_1)+kn\big].&&\\
\end{align}
% \begin{align}
% \nonumber
% (1-e^{2ikd_1})[&-ikn(e^{nd_2}+e^{-nd_2})[e^{ikd_2} + e^{2ik(d_2+d_3)} e^{-ikd_2}]&&\\
% \nonumber
% &+ n^2(e^{nd_2}-e^{-nd_2})[e^{ikd_2} - e^{2ik(d_2+d_3)} e^{-ikd_2}]]&&\\
% \nonumber
% +ik(1+e^{2ik&d_1})[ik(e^{-nd_2}-e^{nd_2})[e^{ikd_2} + e^{2ik(d_2+d_3)} e^{-ikd_2}]&&\\
% \nonumber
% &+n(e^{nd_2}+e^{-nd_2})[e^{ikd_2} - e^{2ik(d_2+d_3)} e^{-ikd_2}]]&&\\
% \end{align}

\end{document}